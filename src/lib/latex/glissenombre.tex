\usepackage{tikz}
\usetikzlibrary{calc,math}
\usepackage{tabularray}
\usepackage{rotating}
\usepackage{makecell}
\usepackage{luacode}
\UseTblrLibrary{tikz,varwidth}
\usepackage{expkv-def}
\usepackage{siunitx}
\usepackage{xstring}
\begin{luacode}
  function numberToArray(nb, animation)
  local result = {}

  -- Initialize array with spaces
  for i = 1, 10 do
  result[i] = " "
  end

  -- Replace comma with dot for easier processing
  local cleanNb = nb:gsub(",", ".")

  -- Apply animation transformation
  local number = tonumber(cleanNb)
  if animation > 0 then
  -- Multiply by 10^animation
  number = number * (10 ^ animation)
  elseif animation < 0 then
  -- Divide by 10^(-animation)
  number = number / (10 ^ (-animation))
  end

  -- Convert back to string and clean up
  local processedNb = tostring(number)

  -- Remove trailing .0 if present (for whole numbers)
  processedNb = processedNb:gsub("%.0$", "")

  -- Find the decimal point position
  local decimalPos = processedNb:find("%.")
  local integerPart, decimalPart

  if decimalPos then
  integerPart = processedNb:sub(1, decimalPos - 1)
  decimalPart = processedNb:sub(decimalPos + 1)
  else
  integerPart = processedNb
  decimalPart = ""
  end

  -- Place integer part (units digit at position 7)
  local unitsPos = 7
  for i = #integerPart, 1, -1 do
  local digit = integerPart:sub(i, i)
  local pos = unitsPos - (#integerPart - i)
  if pos >= 1 and pos <= 10 then
  result[pos] = digit
  end
  end

  -- Place decimal part (starting from position 8)
  for i = 1, #decimalPart do
  local pos = 7 + i
  if pos <= 10 then
  result[pos] = decimalPart:sub(i, i)
  end
  end

  return result, decimalPos
  end

  function operation(nb)
  local op = {
  "\\times \\num{1000000}",
  "\\times \\num{100000}",
  "\\times \\num{10000}",
  "\\times \\num{1000}",
  "\\times 100",
  "\\times 10",
  "",
  "\\div 10",
  "\\div 100",
  "\\div \\num{1000}"
  }
  return op[7-nb]
  end
\end{luacode}
\makeatletter%
% définition des clés pour la commande \glissenombre
\ekvdefinekeys{glissenombre}%
{%
  ,store  animation = \glissenombre@animation%
  ,initial animation = none%
  ,store  color = \glissenombre@color%
  ,initial color = black!20%
  ,invboolTF nocomma = \ifglissenombre@nocomma%
  ,boolTF calcul = \ifglissenombre@calcul%
  ,boolTF zeros = \ifglissenombre@zeros%
}%
\NewDocumentCommand{\glissenombre}{ O{} m }{%
  \begingroup%
  \ekvset{glissenombre}{#1}%
  \luadirect{firstnb,firstcomma = numberToArray(\luastring{#2},0)}%
  \IfInteger{\glissenombre@animation}{%
    \luadirect{secondnb,secondcomma = numberToArray(\luastring{#2},\glissenombre@animation)}%
  }{}%
  \begin{tblrtikzbelow}%
    \IfInteger{\glissenombre@animation}{%
      \tikzmath{%
        integer \x;%
        \x = 7 - \glissenombre@animation;}%
      \fill[\glissenombre@color] (1-\x.north west) rectangle (1-\x.south east);%
      \fill[\glissenombre@color] (3-\x.north west) rectangle (3-\x.south east);%
    }{}%
  \end{tblrtikzbelow}%
  \begin{tblrtikzabove}%
    \foreach \x in {1,...,10} \node at (2-\x) {\luadirect{tex.sprint(firstnb[\x])}};%
    \IfInteger{\glissenombre@animation}{%
      \foreach \x in {1,...,10} \node at (3-\x) {\luadirect{tex.sprint(secondnb[\x])}};%
    }{}%
    \ifglissenombre@nocomma{%
      \luadirect{if firstcomma then tex.sprint("\\node at ([xshift=-5pt,yshift=8pt]2-7.south east) {\\Large,};") end}%
      \IfInteger{\glissenombre@animation}{%
        \luadirect{if secondcomma then tex.sprint("\\node at ([xshift=-5pt,yshift=8pt]3-7.south east) {\\Large,};") end}%
      }{}%
    }{}%
    \IfInteger{\glissenombre@animation}{%
      \ifglissenombre@calcul{%
        \luaexec{tex.sprint(string.format("\\node[above,font=\\tiny] at (1-\%d.north) {$\%s$};",7 - \glissenombre@animation, operation(\glissenombre@animation)))}%
      }{}%
    }{}%
  \end{tblrtikzabove}%
  \renewcommand\cellrotangle{90}%
  \settowidth\rotheadsize{centaine mille}%
  \begin{tblr}{width=\linewidth,colspec={*{10}{X[c,1]}},hlines,vlines,stretch=2%
      ,measure=vbox%
      ,row{1}={cmd={\rotcell},font={\bfseries\footnotesize}}%
    }%
    Millions & Centaines de milliers & Dizaines de milliers & Milliers & Centaines & Dizaines & Unités & Dixièmes & Centièmes & Millièmes \\%
             &                       &                      &          &           &          &        &          &           &           \\%
             &                       &                      &          &           &          &        &          &           &           \\%
  \end{tblr}%
  \endgroup%
}%
\makeatother